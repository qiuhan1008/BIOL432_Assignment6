% Options for packages loaded elsewhere
\PassOptionsToPackage{unicode}{hyperref}
\PassOptionsToPackage{hyphens}{url}
%
\documentclass[
]{article}
\usepackage{amsmath,amssymb}
\usepackage{lmodern}
\usepackage{iftex}
\ifPDFTeX
  \usepackage[T1]{fontenc}
  \usepackage[utf8]{inputenc}
  \usepackage{textcomp} % provide euro and other symbols
\else % if luatex or xetex
  \usepackage{unicode-math}
  \defaultfontfeatures{Scale=MatchLowercase}
  \defaultfontfeatures[\rmfamily]{Ligatures=TeX,Scale=1}
\fi
% Use upquote if available, for straight quotes in verbatim environments
\IfFileExists{upquote.sty}{\usepackage{upquote}}{}
\IfFileExists{microtype.sty}{% use microtype if available
  \usepackage[]{microtype}
  \UseMicrotypeSet[protrusion]{basicmath} % disable protrusion for tt fonts
}{}
\makeatletter
\@ifundefined{KOMAClassName}{% if non-KOMA class
  \IfFileExists{parskip.sty}{%
    \usepackage{parskip}
  }{% else
    \setlength{\parindent}{0pt}
    \setlength{\parskip}{6pt plus 2pt minus 1pt}}
}{% if KOMA class
  \KOMAoptions{parskip=half}}
\makeatother
\usepackage{xcolor}
\usepackage[margin=1in]{geometry}
\usepackage{color}
\usepackage{fancyvrb}
\newcommand{\VerbBar}{|}
\newcommand{\VERB}{\Verb[commandchars=\\\{\}]}
\DefineVerbatimEnvironment{Highlighting}{Verbatim}{commandchars=\\\{\}}
% Add ',fontsize=\small' for more characters per line
\usepackage{framed}
\definecolor{shadecolor}{RGB}{248,248,248}
\newenvironment{Shaded}{\begin{snugshade}}{\end{snugshade}}
\newcommand{\AlertTok}[1]{\textcolor[rgb]{0.94,0.16,0.16}{#1}}
\newcommand{\AnnotationTok}[1]{\textcolor[rgb]{0.56,0.35,0.01}{\textbf{\textit{#1}}}}
\newcommand{\AttributeTok}[1]{\textcolor[rgb]{0.77,0.63,0.00}{#1}}
\newcommand{\BaseNTok}[1]{\textcolor[rgb]{0.00,0.00,0.81}{#1}}
\newcommand{\BuiltInTok}[1]{#1}
\newcommand{\CharTok}[1]{\textcolor[rgb]{0.31,0.60,0.02}{#1}}
\newcommand{\CommentTok}[1]{\textcolor[rgb]{0.56,0.35,0.01}{\textit{#1}}}
\newcommand{\CommentVarTok}[1]{\textcolor[rgb]{0.56,0.35,0.01}{\textbf{\textit{#1}}}}
\newcommand{\ConstantTok}[1]{\textcolor[rgb]{0.00,0.00,0.00}{#1}}
\newcommand{\ControlFlowTok}[1]{\textcolor[rgb]{0.13,0.29,0.53}{\textbf{#1}}}
\newcommand{\DataTypeTok}[1]{\textcolor[rgb]{0.13,0.29,0.53}{#1}}
\newcommand{\DecValTok}[1]{\textcolor[rgb]{0.00,0.00,0.81}{#1}}
\newcommand{\DocumentationTok}[1]{\textcolor[rgb]{0.56,0.35,0.01}{\textbf{\textit{#1}}}}
\newcommand{\ErrorTok}[1]{\textcolor[rgb]{0.64,0.00,0.00}{\textbf{#1}}}
\newcommand{\ExtensionTok}[1]{#1}
\newcommand{\FloatTok}[1]{\textcolor[rgb]{0.00,0.00,0.81}{#1}}
\newcommand{\FunctionTok}[1]{\textcolor[rgb]{0.00,0.00,0.00}{#1}}
\newcommand{\ImportTok}[1]{#1}
\newcommand{\InformationTok}[1]{\textcolor[rgb]{0.56,0.35,0.01}{\textbf{\textit{#1}}}}
\newcommand{\KeywordTok}[1]{\textcolor[rgb]{0.13,0.29,0.53}{\textbf{#1}}}
\newcommand{\NormalTok}[1]{#1}
\newcommand{\OperatorTok}[1]{\textcolor[rgb]{0.81,0.36,0.00}{\textbf{#1}}}
\newcommand{\OtherTok}[1]{\textcolor[rgb]{0.56,0.35,0.01}{#1}}
\newcommand{\PreprocessorTok}[1]{\textcolor[rgb]{0.56,0.35,0.01}{\textit{#1}}}
\newcommand{\RegionMarkerTok}[1]{#1}
\newcommand{\SpecialCharTok}[1]{\textcolor[rgb]{0.00,0.00,0.00}{#1}}
\newcommand{\SpecialStringTok}[1]{\textcolor[rgb]{0.31,0.60,0.02}{#1}}
\newcommand{\StringTok}[1]{\textcolor[rgb]{0.31,0.60,0.02}{#1}}
\newcommand{\VariableTok}[1]{\textcolor[rgb]{0.00,0.00,0.00}{#1}}
\newcommand{\VerbatimStringTok}[1]{\textcolor[rgb]{0.31,0.60,0.02}{#1}}
\newcommand{\WarningTok}[1]{\textcolor[rgb]{0.56,0.35,0.01}{\textbf{\textit{#1}}}}
\usepackage{graphicx}
\makeatletter
\def\maxwidth{\ifdim\Gin@nat@width>\linewidth\linewidth\else\Gin@nat@width\fi}
\def\maxheight{\ifdim\Gin@nat@height>\textheight\textheight\else\Gin@nat@height\fi}
\makeatother
% Scale images if necessary, so that they will not overflow the page
% margins by default, and it is still possible to overwrite the defaults
% using explicit options in \includegraphics[width, height, ...]{}
\setkeys{Gin}{width=\maxwidth,height=\maxheight,keepaspectratio}
% Set default figure placement to htbp
\makeatletter
\def\fps@figure{htbp}
\makeatother
\setlength{\emergencystretch}{3em} % prevent overfull lines
\providecommand{\tightlist}{%
  \setlength{\itemsep}{0pt}\setlength{\parskip}{0pt}}
\setcounter{secnumdepth}{-\maxdimen} % remove section numbering
\ifLuaTeX
  \usepackage{selnolig}  % disable illegal ligatures
\fi
\IfFileExists{bookmark.sty}{\usepackage{bookmark}}{\usepackage{hyperref}}
\IfFileExists{xurl.sty}{\usepackage{xurl}}{} % add URL line breaks if available
\urlstyle{same} % disable monospaced font for URLs
\hypersetup{
  pdftitle={Analysis},
  pdfauthor={Qiuhan Zhang},
  hidelinks,
  pdfcreator={LaTeX via pandoc}}

\title{Analysis}
\author{Qiuhan Zhang}
\date{2023-02-14}

\begin{document}
\maketitle

\begin{Shaded}
\begin{Highlighting}[]
\CommentTok{\#load library}
\FunctionTok{library}\NormalTok{(readr)}
\FunctionTok{library}\NormalTok{(dplyr)}
\end{Highlighting}
\end{Shaded}

\begin{verbatim}
## 
## Attaching package: 'dplyr'
\end{verbatim}

\begin{verbatim}
## The following objects are masked from 'package:stats':
## 
##     filter, lag
\end{verbatim}

\begin{verbatim}
## The following objects are masked from 'package:base':
## 
##     intersect, setdiff, setequal, union
\end{verbatim}

\begin{Shaded}
\begin{Highlighting}[]
\FunctionTok{library}\NormalTok{(knitr)}
\end{Highlighting}
\end{Shaded}

\begin{Shaded}
\begin{Highlighting}[]
\CommentTok{\#Import the Sequences.csv file.}
\NormalTok{Sequences }\OtherTok{\textless{}{-}} \FunctionTok{read\_csv}\NormalTok{(}\StringTok{"Sequences.csv"}\NormalTok{)}
\end{Highlighting}
\end{Shaded}

\begin{verbatim}
## New names:
## Rows: 3 Columns: 3
## -- Column specification
## -------------------------------------------------------- Delimiter: "," chr
## (2): Name, Sequence dbl (1): ...1
## i Use `spec()` to retrieve the full column specification for this data. i
## Specify the column types or set `show_col_types = FALSE` to quiet this message.
## * `` -> `...1`
\end{verbatim}

\#\#Count the number of each base pair (A, T, C and G), in each of the
three sequences.

\begin{Shaded}
\begin{Highlighting}[]
\CommentTok{\#convert to characters}
\NormalTok{seq\_1 }\OtherTok{\textless{}{-}} \FunctionTok{as.character}\NormalTok{(Sequences}\SpecialCharTok{$}\NormalTok{Sequence[}\DecValTok{1}\NormalTok{])}
\NormalTok{seq\_2 }\OtherTok{\textless{}{-}} \FunctionTok{as.character}\NormalTok{(Sequences}\SpecialCharTok{$}\NormalTok{Sequence[}\DecValTok{2}\NormalTok{])}
\NormalTok{seq\_3 }\OtherTok{\textless{}{-}} \FunctionTok{as.character}\NormalTok{(Sequences}\SpecialCharTok{$}\NormalTok{Sequence[}\DecValTok{3}\NormalTok{])}
\end{Highlighting}
\end{Shaded}

\begin{Shaded}
\begin{Highlighting}[]
\CommentTok{\#count the number of each base pair}
\DocumentationTok{\#\#sequence 1}
\NormalTok{count1\_A }\OtherTok{\textless{}{-}} \FunctionTok{nchar}\NormalTok{(}\FunctionTok{gsub}\NormalTok{(}\StringTok{"[\^{}A]"}\NormalTok{, }\StringTok{""}\NormalTok{, seq\_1))}
\NormalTok{count1\_T }\OtherTok{\textless{}{-}} \FunctionTok{nchar}\NormalTok{(}\FunctionTok{gsub}\NormalTok{(}\StringTok{"[\^{}T]"}\NormalTok{, }\StringTok{""}\NormalTok{, seq\_1))}
\NormalTok{count1\_C }\OtherTok{\textless{}{-}} \FunctionTok{nchar}\NormalTok{(}\FunctionTok{gsub}\NormalTok{(}\StringTok{"[\^{}C]"}\NormalTok{, }\StringTok{""}\NormalTok{, seq\_1))}
\NormalTok{count1\_G }\OtherTok{\textless{}{-}} \FunctionTok{nchar}\NormalTok{(}\FunctionTok{gsub}\NormalTok{(}\StringTok{"[\^{}G]"}\NormalTok{, }\StringTok{""}\NormalTok{, seq\_1))}

\DocumentationTok{\#\#sequence 2}
\NormalTok{count2\_A }\OtherTok{\textless{}{-}} \FunctionTok{nchar}\NormalTok{(}\FunctionTok{gsub}\NormalTok{(}\StringTok{"[\^{}A]"}\NormalTok{, }\StringTok{""}\NormalTok{, seq\_2))}
\NormalTok{count2\_T }\OtherTok{\textless{}{-}} \FunctionTok{nchar}\NormalTok{(}\FunctionTok{gsub}\NormalTok{(}\StringTok{"[\^{}T]"}\NormalTok{, }\StringTok{""}\NormalTok{, seq\_2))}
\NormalTok{count2\_C }\OtherTok{\textless{}{-}} \FunctionTok{nchar}\NormalTok{(}\FunctionTok{gsub}\NormalTok{(}\StringTok{"[\^{}C]"}\NormalTok{, }\StringTok{""}\NormalTok{, seq\_2))}
\NormalTok{count2\_G }\OtherTok{\textless{}{-}} \FunctionTok{nchar}\NormalTok{(}\FunctionTok{gsub}\NormalTok{(}\StringTok{"[\^{}G]"}\NormalTok{, }\StringTok{""}\NormalTok{, seq\_2))}

\DocumentationTok{\#\#sequence 3}
\NormalTok{count3\_A }\OtherTok{\textless{}{-}} \FunctionTok{nchar}\NormalTok{(}\FunctionTok{gsub}\NormalTok{(}\StringTok{"[\^{}A]"}\NormalTok{, }\StringTok{""}\NormalTok{, seq\_3))}
\NormalTok{count3\_T }\OtherTok{\textless{}{-}} \FunctionTok{nchar}\NormalTok{(}\FunctionTok{gsub}\NormalTok{(}\StringTok{"[\^{}T]"}\NormalTok{, }\StringTok{""}\NormalTok{, seq\_3))}
\NormalTok{count3\_C }\OtherTok{\textless{}{-}} \FunctionTok{nchar}\NormalTok{(}\FunctionTok{gsub}\NormalTok{(}\StringTok{"[\^{}C]"}\NormalTok{, }\StringTok{""}\NormalTok{, seq\_3))}
\NormalTok{count3\_G }\OtherTok{\textless{}{-}} \FunctionTok{nchar}\NormalTok{(}\FunctionTok{gsub}\NormalTok{(}\StringTok{"[\^{}G]"}\NormalTok{, }\StringTok{""}\NormalTok{, seq\_3))}
\end{Highlighting}
\end{Shaded}

\begin{Shaded}
\begin{Highlighting}[]
\CommentTok{\#Print out each sequence.}
\FunctionTok{print}\NormalTok{(seq\_1)}
\end{Highlighting}
\end{Shaded}

\begin{verbatim}
## [1] "AGCATGCAAGTCAAACGAGATGTAGCAATACATCTAGTGGCGAACGGGTGAGTAACGCGTGGATGATCTACCTATGAGATGGGGATAACTATTAGAAATAGTAGCTAATACCGAATAAGGTCAATTAATTTGTTAATTGATGAAAGGAAGCCTTTAAAGCTTCGCTTGTAGATGAGTCTGCGTCTTATTAGTTAGTTGGTAGGGTAAATGCCTACCAAGGCGATGATAAGTAACCGGCCTGAGAGGGTGAACGGTCACACTGGAACTGAGACACGGTCCAGACTCCTACGGGAGGCAGCAGCTAAGAATCTTCCGCAATGGGCGAAAGCCTGACGGAGCGACACTGCGTGAATGAAGAAGGTCGAAAGATTGTAAAATTCTTTTATAAATGAGGAATAAGCTTTGTAGGAAATGACGAAGTGATGACGTTAATTTATGAATAAGCCCCGGCTAATTACGTGCCAGCAGCCGCGGTAATACG"
\end{verbatim}

\begin{Shaded}
\begin{Highlighting}[]
\FunctionTok{print}\NormalTok{(seq\_2)}
\end{Highlighting}
\end{Shaded}

\begin{verbatim}
## [1] "AGCATGCAAGTCAAACGGGATGTAGCAATACATTCAGTGGCGAACGGGTGAGTAACGCGTGGATGATCTACCTATGAGATGGGGATAACTATTAGAAATAGTAGCTAATACCGAATAAGGTCAGTTAATTTGTTAATTGATGAAAGGAAGCCTTTAAAGCTTCGCTTGTAGATGAGTCTGCGTCTTATTAGCTAGTTGGTAGGGTAAATGCCTACCAAGGCAATGATAAGTAACCGGCCTGAGAGGGTGAACGGTCACACTGGAACTGAGATACGGTCCAGACTCCTACGGGAGGCAGCAGCTAAGAATCTTCCGCAATGGGCGAAAGCCTGACGGAGCGACACTGCGTGAATGAAGAAGGTCGAAAGATTGTAAAATTCTTTTATAAATGAGGAATAAGCTTTGTAGGAAATGACAAAGTGATGACGTTAATTTATGAATAAGCCCCGGCTAATTACGTGCCAGCAGCAGCGGTAATACG"
\end{verbatim}

\begin{Shaded}
\begin{Highlighting}[]
\FunctionTok{print}\NormalTok{(seq\_3)}
\end{Highlighting}
\end{Shaded}

\begin{verbatim}
## [1] "AGCATGCAAGTCAAACGAGATGTAGTAATACATCTAGTGGCGAACGGGTGAGTAACGCGTGGATGATCTACCTATGAGATGGGGATAACTATTAGAAATAGTAGCTAATACCGAATAAGGTCAATTAATTTGTTAATTGATGAAAGGAAGCCTTTAAAGCTTCGCTTGTAGATGAGTCTGCGTCTTATTAGTTAGTTGGTAGGGTAAATGCCTACCAAGGCGATGATAAGTAACCGGCCTGAGAGGGTGAACGGTCACACTGGAACTGAGACACGGTCCAGACTCCTACGGGAGGCAGCAGCTAAGAATCTTCCGCAATGGGCGAAAGCCTGACGGAGCGACACTGCGTGAATGAAGAAGGTCGAAAGATTGTAAAATTCTTTTATAAATGAGGAATAAGCTTTGTAGGAAATGACGAAGTGATGACGTTAATTTATGAATAAGCCCCGGCTAATTACGTGCCAGCAGCCGCGGTAATACG"
\end{verbatim}

\begin{Shaded}
\begin{Highlighting}[]
\CommentTok{\#Print out the number of each nucleotide as a table for each of the three sequences.}
\NormalTok{results }\OtherTok{\textless{}{-}} \FunctionTok{data.frame}\NormalTok{(}\AttributeTok{Sequence\_Name =} \FunctionTok{c}\NormalTok{(}\StringTok{"HQ433692.1"}\NormalTok{, }\StringTok{"HQ433694."}\NormalTok{, }\StringTok{"HQ433691.1"}\NormalTok{),}
                          \AttributeTok{A =} \FunctionTok{c}\NormalTok{(count1\_A, count2\_A, count3\_A),}
                          \AttributeTok{T =} \FunctionTok{c}\NormalTok{(count1\_T, count2\_T, count3\_T),}
                          \AttributeTok{C =} \FunctionTok{c}\NormalTok{(count1\_C, count2\_C, count3\_C),}
                          \AttributeTok{G =} \FunctionTok{c}\NormalTok{(count1\_G, count2\_G, count3\_G))}
\FunctionTok{print}\NormalTok{(results)}
\end{Highlighting}
\end{Shaded}

\begin{verbatim}
##   Sequence_Name   A   T  C   G
## 1    HQ433692.1 154 114 82 131
## 2     HQ433694. 155 114 81 131
## 3    HQ433691.1 154 115 81 131
\end{verbatim}

\#\#Include an image of a bacteria from the internet, and a link to the
Wikipedia page about Borrelia burgdorferi

\href{PIXNIO-38518-4252x2890.jpeg}{Photo of Borrelia burgdorferi}

\href{https://en.wikipedia.org/wiki/Borrelia_burgdorferi}{Borrelia
burgdorferi wikipedia link}

\#\#Calculate GC Content (\% of nucleotides that are G or C) and create
a final table showing GC content for each sequence ID

\begin{Shaded}
\begin{Highlighting}[]
\NormalTok{GC\_count }\OtherTok{\textless{}{-}}\NormalTok{ results }\SpecialCharTok{\%\textgreater{}\%}
  \FunctionTok{group\_by}\NormalTok{(Sequence\_Name) }\SpecialCharTok{\%\textgreater{}\%}
  \FunctionTok{mutate}\NormalTok{(}\AttributeTok{GC\_count =}\NormalTok{ ((C }\SpecialCharTok{+}\NormalTok{G) }\SpecialCharTok{/}\NormalTok{ (A }\SpecialCharTok{+}\NormalTok{ T }\SpecialCharTok{+}\NormalTok{ C }\SpecialCharTok{+}\NormalTok{ G)) }\SpecialCharTok{*} \DecValTok{100}\NormalTok{) }\SpecialCharTok{\%\textgreater{}\%}
  \FunctionTok{select}\NormalTok{(Sequence\_Name, GC\_count)}
\NormalTok{GC\_count}
\end{Highlighting}
\end{Shaded}

\begin{verbatim}
## # A tibble: 3 x 2
## # Groups:   Sequence_Name [3]
##   Sequence_Name GC_count
##   <chr>            <dbl>
## 1 HQ433692.1        44.3
## 2 HQ433694.         44.1
## 3 HQ433691.1        44.1
\end{verbatim}

\begin{Shaded}
\begin{Highlighting}[]
\NormalTok{data1 }\OtherTok{\textless{}{-}} \FunctionTok{read\_csv}\NormalTok{(}\StringTok{"\textasciitilde{}/Desktop/data1.csv"}\NormalTok{)}
\end{Highlighting}
\end{Shaded}

\begin{verbatim}
## Rows: 3 Columns: 2
## -- Column specification --------------------------------------------------------
## Delimiter: ","
## chr (2): Sequence_Name, GC_count
## 
## i Use `spec()` to retrieve the full column specification for this data.
## i Specify the column types or set `show_col_types = FALSE` to quiet this message.
\end{verbatim}

\begin{Shaded}
\begin{Highlighting}[]
\NormalTok{table1 }\OtherTok{\textless{}{-}}\NormalTok{ data1 }\SpecialCharTok{\%\textgreater{}\%}
  \FunctionTok{select}\NormalTok{(Sequence\_Name, GC\_count)}
\NormalTok{table1}
\end{Highlighting}
\end{Shaded}

\begin{verbatim}
## # A tibble: 3 x 2
##   Sequence_Name GC_count
##   <chr>         <chr>   
## 1 HQ433692.1    44.28%  
## 2 HQ433694.     44.07%  
## 3 HQ433694.     44.07%
\end{verbatim}

\end{document}
